% --------------------------------------------------------------
%                         Template HW
% --------------------------------------------------------------
\documentclass[a4paper, 10pt]{article} %draft = show box warnings
\usepackage[utf8]{inputenc} % Accept different input encodings [utf8]
%\usepackage[T1]{fontenc}    % Standard package for selecting font encodings
%\usepackage[a4paper, total={6.5in,10.2in}]{geometry} % Flexible and complete interface to document dimensions
\renewcommand{\baselinestretch}{1.3} 

% --------------------------------------------------------------
%                       Packages
% --------------------------------------------------------------
\usepackage{graphicx} % Import images
\usepackage[linktoc=all]{hyperref} % Create hyperlinks
\usepackage{float} % Good placement for float objects
\usepackage[export]{adjustbox} % Positioning figures
%\usepackage{amsmath,amsthm,amssymb} % American Mathematics Society facilities
\usepackage[fancysections,titlepage,sectionmark]{polytechnique}

% --------------------------------------------------------------
%                         Languages
%			Change also the exercise environment
% --------------------------------------------------------------
%\usepackage[english]{babel} % Multilingual support [english]
\usepackage[french]{babel} % Multilingual support [french]
%\usepackage[brazilian]{babel} % Multilingual support [pt-BR]

% --------------------------------------------------------------
%                         Fonts
% --------------------------------------------------------------
%\usepackage{lmodern} % Good looking T1 font
%\usepackage{mathpazo} % Hermann Zapf's Palatino font
%\usepackage{kpfonts} % Kepler font
%\usepackage{mathptmx} % Times New Roman Like Font
%\usepackage{eulervm} %  AMS Euler (eulervm) math font.

% --------------------------------------------------------------
%                       Begin Document
% --------------------------------------------------------------

\begin{document}

\title{Rapport de stage 2A}
\subtitle{Thales Research \& Technology}
\author{
\begin{tabular}{ccc}
	\textbf{Stagiaire}
	\\
	Alexandre RIBEIRO JOÃO MACEDO
	\\	
	\textbf{Tuteur en entreprise} 
	\\
	M. Jean-Marc MOTA
%	\\
%	\textbf{Référent AX}
%	\\
%	M. Olivier DELASSUS
\end{tabular}
%\logo{polytechnique-logohori.eps}
}

\maketitle

\section{Executive Summary}
Le stage de découverte d'entreprise fait comme une composant obligatoire du curriculum de deuxième année du cycle d'ingénieur polytechnicien est essentielle à formation du futur ingénieur. Dans le cadre de ce stage, l'élève a l'opportunité de développer un travail pour une entreprise et de découvrir le monde du travail. Ce qui est bien différente de celui de l'école.

Mon stage a été fait à Thales Recherche et Technologie dans le Laboratoire de Systèmes Embarqué Critique, où j'ai pu faire un projet dans le domaine de méthodes formelles. Plus spécifiquement, j'ai travaillé avec l'écriture des spécifications formelles des algorithmes distribués, leur \textit{Model Checking} et l'implémentation d'un visualisateur pour les résultats de ce dernier. Le \textit{Model Checking} 

\newpage
\section{Remerciements}
Je voudrais remercie M. Jean-Marc Mota mon tuteur responsable par mon stage

\newpage
\tableofcontents

\newpage
\section{Introduction}

\newpage
\section{Présentation de l'entreprise}

\newpage
\section{Travail technique}

\newpage
\section{Enseignement et apports du stage}
\subsection{Forum des stagiaires}
\end{document}
