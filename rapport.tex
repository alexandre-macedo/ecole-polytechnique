% --------------------------------------------------------------
%                         Template HW
% --------------------------------------------------------------
\documentclass[a4paper, 12pt]{article} %draft = show box warnings
\usepackage[utf8]{inputenc} % Accept different input encodings [utf8]
%\usepackage[T1]{fontenc}    % Standard package for selecting font encodings
%\usepackage[a4paper, total={6.5in,10.2in}]{geometry} % Flexible and complete interface to document dimensions
\renewcommand{\baselinestretch}{1.3} 

% --------------------------------------------------------------
%                       Packages
% --------------------------------------------------------------
\usepackage{graphicx} % Import images
\usepackage[linktoc=all]{hyperref} % Create hyperlinks
\usepackage{float} % Good placement for float objects
\usepackage[export]{adjustbox} % Positioning figures
%\usepackage{amsmath,amsthm,amssymb} % American Mathematics Society facilities
\usepackage[fancysections,titlepage,sectionmark]{polytechnique}

% --------------------------------------------------------------
%                         Languages
%			Change also the exercise environment
% --------------------------------------------------------------
%\usepackage[english]{babel} % Multilingual support [english]
\usepackage[french]{babel} % Multilingual support [french]
%\usepackage[brazilian]{babel} % Multilingual support [pt-BR]

% --------------------------------------------------------------
%                         Fonts
% --------------------------------------------------------------
%\usepackage{lmodern} % Good looking T1 font
%\usepackage{mathpazo} % Hermann Zapf's Palatino font
%\usepackage{kpfonts} % Kepler font
%\usepackage{mathptmx} % Times New Roman Like Font
%\usepackage{eulervm} %  AMS Euler (eulervm) math font.

% --------------------------------------------------------------
%                       Begin Document
% --------------------------------------------------------------

\begin{document}

\title{Rapport de stage en entreprise}
\subtitle{Thales Research \& Technology}
\author{
\begin{tabular}{ccc}
	\textbf{Stagiaire}
	\\
	Alexandre RIBEIRO JOÃO MACEDO
	\\	
	\textbf{Tuteur en entreprise} 
	\\
	M. Jean-Marc MOTA
	\\
	\textbf{Référent AX}
	\\
	M. Olivier DELASSUS
\end{tabular}
}
\date{19 septembre 2017}
\maketitle

\section{Executive Summary}
Le stage de découverte d'entreprise fait comme une composant obligatoire du curriculum de deuxième année du cycle d'ingénieur polytechnicien est essentielle à formation du futur ingénieur. Dans le cadre de ce stage, l'élève a l'opportunité de développer un travail pour une entreprise et de découvrir le monde du travail. Ce qui est bien différente de celui de l'école.

Mon stage a été fait à Thales Recherche et Technologie dans le Laboratoire de Systèmes Embarqué Critique, où j'ai pu faire un projet dans le domaine de méthodes formelles. Plus spécifiquement, j'ai travaillé avec l'écriture des spécifications formelles des algorithmes distribués, leur \textit{Model Checking} et l'implémentation d'une visualisation pour les résultats de ce dernier. Le \textit{Model Checking} est important pour trouver les erreurs logiques dans les algorithmes qui sont difficile à prévoir pendant sa conception et qui peuvent être dangereuses s'ils sont utilisent dans la production.

Pendant mon stage de douze semaines, de 12 juillet 2017 à 31 août 2017, j'ai été orientée par M. Jean-Marc Mota qui a beaucoup m'aidé dans le côté technique, mais qui a aussi me donné des conseilles pour la vie professionnelle. Un autre expérience intéressant pendant le stage a été le Forum des Stagiaires de Thales, où les stagiaires ont été donnés des conseilles pour la réussit professionnelle.

Le but de ce rapport est de décrire les expérience vécues pendants ces douze semaines et le apport de ces expériences dans ma formation.

\newpage
\section{Remerciements}
Je voudrais remercie M. Jean-Marc Mota mon tuteur responsable par mon stage

\newpage
\tableofcontents

\newpage
\section{Introduction}
La révolution du numérique est un phénomène qui a eu leur place pendant les trente dernières années et, à chaque jour, les systèmes informatiques ont plus d'influence et responsabilité sur la vie des personnes. Alors, il n'est pas incroyable que beaucoup de monde ayant un intérêt par ce domaine.  

\newpage
\section{Présentation de l'entreprise}
\subsection{Chiffres d'affaires et part du marché}
\subsection{Forum des stagiaires}

\newpage
\section{Travail technique}

\newpage
\section{Enseignement et apports du stage}


\newpage
\section{Conclusions}
\end{document}
