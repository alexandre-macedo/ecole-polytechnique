% --------------------------------------------------------------
%                         Template HW
% --------------------------------------------------------------
\documentclass[10pt]{article} %draft = show box warnings
\usepackage[utf8]{inputenc} % Accept different input encodings [utf8]
\usepackage[T1]{fontenc}    % Standard package for selecting font encodings
\usepackage[a4paper, total={6.5in,10.2in}]{geometry} % Flexible and complete interface to document dimensions
\renewcommand{\baselinestretch}{1.3} 

% --------------------------------------------------------------
%                       Packages
% --------------------------------------------------------------
\usepackage{graphicx} % Import images
\usepackage[linktoc=all]{hyperref} % Create hyperlinks
\usepackage{float} % Good placement for float objects
\usepackage[export]{adjustbox} % Positioning figures
\usepackage{amsmath,amsthm,amssymb} % American Mathematics Society facilities


% --------------------------------------------------------------
%                       Header Definition
% --------------------------------------------------------------
\newcommand{\createHeader}[6]{
	\noindent
	\normalsize\textbf{#2} \hfill \textbf{#1}\\
	\normalsize\textbf{#3} \hfill \textbf{\displaydate{#6}}\vspace{20pt}
	\centerline{\Large \textbf{#5}}\vspace{3pt}
	\centerline{\normalsize #4}\vspace{20pt}}


% --------------------------------------------------------------
%                         Languages
%			Change also the exercise environment
% --------------------------------------------------------------
%\usepackage[english]{babel} % Multilingual support [english]
\usepackage[french]{babel} % Multilingual support [french]
%\usepackage[brazilian]{babel} % Multilingual support [pt-BR]

\usepackage{datetime}

% --------------------------------------------------------------
%                         Fonts
% --------------------------------------------------------------
\usepackage{lmodern} % Good looking T1 font
%\usepackage{mathpazo} % Hermann Zapf's Palatino font
%\usepackage{kpfonts} % Kepler font
%\usepackage{mathptmx} % Times New Roman Like Font
%\usepackage{eulervm} %  AMS Euler (eulervm) math font.

% --------------------------------------------------------------
%                       Begin Document
% --------------------------------------------------------------

\begin{document}
\newdate{date}{11}{04}{2017}
\createHeader{École Polytechnique}{Atelier d'écriture}{LAN484fFLE}{Alexandre Ribeiro João Macedo}{Cours 1 - Réalisme}{date}

Lucas Renault est sorti du Restaurant Le Métro, dans le cinquième arrondissement, à 15 heures un dimanche. Il faisait chaud comme d'habitude pour l'été parisien. M. Renault était habillé avec un costume noir et il transpirait après un gros déjeuner qu'il venait de manger. Il avait à peu près soixante ans, les chevaux blanc et de petits yeux noirs. 

Il a habité à Paris pendant toute sa vie, sauf pendant six mois où il a habité en Belgique quand il avait 17 ans. Un période qu'il n'a pas aimé. Il aimait Paris et il ne sortait jamais de la ville pour un période plus long que quatre jours. Grace à cet amour, il aimait aller au cinquième arrondissement les dimanches pour voir les touristes qui prenaient des photos de sa ville préféré. Il avait un rituel où il se réveillait à huit heures de la matin, il sortait de son appartement au première arrondissement, il allait à sa boulangerie préféré pour prendre le petit-déjeuner et lire le journal. Et l'après-midi il se habillait avec son costume de dimanche et il se promenait par la rue en regardant les touristes qu'il beaucoup aimait jusqu'au moment où il avait faim quand il choisissait un restaurant pour le déjeuner.

Le jour où il est sorti du Le Métro n'avait rien d'spécial.



\end{document}
