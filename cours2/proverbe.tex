% --------------------------------------------------------------
%                         Template HW
% --------------------------------------------------------------
\documentclass[10pt]{article} %draft = show box warnings
\usepackage[utf8]{inputenc} % Accept different input encodings [utf8]
\usepackage[T1]{fontenc}    % Standard package for selecting font encodings
\usepackage[a4paper, total={6.5in,10.2in}]{geometry} % Flexible and complete interface to document dimensions
\renewcommand{\baselinestretch}{1.3} 

% --------------------------------------------------------------
%                       Packages
% --------------------------------------------------------------
\usepackage{graphicx} % Import images
\usepackage[linktoc=all]{hyperref} % Create hyperlinks
\usepackage{float} % Good placement for float objects
\usepackage[export]{adjustbox} % Positioning figures
\usepackage{amsmath,amsthm,amssymb} % American Mathematics Society facilities


% --------------------------------------------------------------
%                       Header Definition
% --------------------------------------------------------------
\newcommand{\createHeader}[6]{
	\noindent
	\normalsize\textbf{#2} \hfill \textbf{#1}\\
	\normalsize\textbf{#3} \hfill \textbf{\displaydate{#6}}\vspace{20pt}
	\centerline{\Large \textbf{#5}}\vspace{3pt}
	\centerline{\normalsize #4}\vspace{20pt}}


% --------------------------------------------------------------
%                         Languages
%			Change also the exercise environment
% --------------------------------------------------------------
%\usepackage[english]{babel} % Multilingual support [english]
\usepackage[french]{babel} % Multilingual support [french]
%\usepackage[brazilian]{babel} % Multilingual support [pt-BR]

\usepackage{datetime}

% --------------------------------------------------------------
%                         Fonts
% --------------------------------------------------------------
\usepackage{lmodern} % Good looking T1 font
%\usepackage{mathpazo} % Hermann Zapf's Palatino font
%\usepackage{kpfonts} % Kepler font
%\usepackage{mathptmx} % Times New Roman Like Font
%\usepackage{eulervm} %  AMS Euler (eulervm) math font.

% --------------------------------------------------------------
%                       Begin Document
% --------------------------------------------------------------

\begin{document}
	\newdate{date}{18}{04}{2017}
	\createHeader{École Polytechnique}{Atelier d'écriture}{LAN484fFLE}{Alexandre Ribeiro João Macedo}{Cours 2 - Proverbe}{date}
	Proverbes:
	\begin{itemize}
		\item Il faut battre le fer quand il est chaud. (Internationale)
		\\
		\item C'est mieux d'avoir un oiseau que deux oiseaux qui volent. (Brésil)
		\item C'est mieux d'avoir un oiseau que deux oiseaux dans la forêt. (Chine)
		\item Un "tiens!" vaut mieux que deux "tu l'auras". (France)
		\\
		\item On compte les poulets en automne. (Russe)
		\item On ne vends pas la peau de l'ours avant de l'avoir tué. (France)
		\\
		\item Un marchand de bon vin ne s'inquiète pas de la place de sa boutique. (Chine)
		\\
		\item Le meilleur moment d'une journée est le matin, le meilleur moment d'une année est le printemps. (Chine)
		\item En mai fait ce qu'il te plaît. (France)
		\\
		\item Avoir un poil dan las main. (=être paresseux) (France)
		\item Avoir un poutre dan las main. (=être paresseux) (France)
		\\
		\item On ne peut avoir du poisson et la patte de l'ours en même temps. (Chine)
		\item On ne peut avoir le beurre, l'argent du beurre, et le sourire de la crémière. (France) 
		\\
		\item Ça ne mange pas de pain. (France)
		\\
		\item Après la pluie, le beau temps. (France)
	\end{itemize}
\end{document}
