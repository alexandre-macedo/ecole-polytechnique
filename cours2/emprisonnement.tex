% --------------------------------------------------------------
%                         Template HW
% --------------------------------------------------------------
\documentclass[10pt]{article} %draft = show box warnings
\usepackage[utf8]{inputenc} % Accept different input encodings [utf8]
\usepackage[T1]{fontenc}    % Standard package for selecting font encodings
\usepackage[a4paper, total={6.5in,10.2in}]{geometry} % Flexible and complete interface to document dimensions
\renewcommand{\baselinestretch}{1.3} 

% --------------------------------------------------------------
%                       Packages
% --------------------------------------------------------------
\usepackage{graphicx} % Import images
\usepackage[linktoc=all]{hyperref} % Create hyperlinks
\usepackage{float} % Good placement for float objects
\usepackage[export]{adjustbox} % Positioning figures
\usepackage{amsmath,amsthm,amssymb} % American Mathematics Society facilities


% --------------------------------------------------------------
%                       Header Definition
% --------------------------------------------------------------
\newcommand{\createHeader}[6]{
	\noindent
	\normalsize\textbf{#2} \hfill \textbf{#1}\\
	\normalsize\textbf{#3} \hfill \textbf{\displaydate{#6}}\vspace{20pt}
	\centerline{\Large \textbf{#5}}\vspace{3pt}
	\centerline{\normalsize #4}\vspace{20pt}}


% --------------------------------------------------------------
%                         Languages
%			Change also the exercise environment
% --------------------------------------------------------------
%\usepackage[english]{babel} % Multilingual support [english]
\usepackage[french]{babel} % Multilingual support [french]
%\usepackage[brazilian]{babel} % Multilingual support [pt-BR]

\usepackage{datetime}

% --------------------------------------------------------------
%                         Fonts
% --------------------------------------------------------------
\usepackage{lmodern} % Good looking T1 font
%\usepackage{mathpazo} % Hermann Zapf's Palatino font
%\usepackage{kpfonts} % Kepler font
%\usepackage{mathptmx} % Times New Roman Like Font
%\usepackage{eulervm} %  AMS Euler (eulervm) math font.

% --------------------------------------------------------------
%                       Begin Document
% --------------------------------------------------------------

\begin{document}
	\newdate{date}{18}{04}{2017}
	\createHeader{École Polytechnique}{Atelier d'écriture}{LAN484fFLE}{Alexandre Ribeiro João Macedo}{Cours 2 - Emprisonnement}{date}
	
Quand je suis arrivé en prison, le gardien, qui n'était pas très gentil, m'a demandé de changer mes vêtements pour un salopette orange et des bottes noires. Il m'a demandé aussi de déposer toutes mes affaires dans une petite boite. Je n'avais pas grand chose avec moi : une montre et des écouteurs, deux choses dont je n’avais plus besoin. Le plus remarquable de cette petite salle où tous les condamnés devraient passer avant d'aller a ses cellules était l'odeur, une mélange de sueur et de moule terribles.

Après cette petite introduction à la vie dans la prison, j'ai été ramené à la cellule où j'ai trouvé un petit lit et une peinture très jolie sur le mur. Dans cette peinture il y avait la première maison où j'ai habité, mais elle était à la campagne juste à coté d’un bois. Cette nuit, je me suis couché en pensant à cette petite maison transposée. J'ai été très surpris le matin quand je me suis réveillé et la peinture a été remplacé par une peinture de chien. En plus, la porte de la cellule était ouverte. J'ai été encore plus surpris quand j'ai vu que j'étais le seule sur la prison. J'ai marché sur le terrain jusqu'au moment où j'ai entendu mon nom. Quand je me suis retourné, je me suis réveillé sur mon lit, pas dans la prison, mais chez ma grand mère. C'était l'heure du déjeuner.   
\end{document}
