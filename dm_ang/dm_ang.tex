% --------------------------------------------------------------
%                         Template HW
% --------------------------------------------------------------
\documentclass[12pt]{article} %draft = show box warnings
\usepackage[utf8]{inputenc} % Accept different input encodings [utf8]
\usepackage[T1]{fontenc}    % Standard package for selecting font encodings
\usepackage[a4paper, total={6.5in,10.2in}]{geometry} % Flexible and complete interface to document dimensions
\renewcommand{\baselinestretch}{1.3} 
\usepackage{lineno}

% --------------------------------------------------------------
%                       Packages
% --------------------------------------------------------------
\usepackage{graphicx} % Import images
\usepackage[linktoc=all]{hyperref} % Create hyperlinks
\hypersetup{
	colorlinks=true,
	linkcolor=blue,
	filecolor=magenta,      
	urlcolor=cyan,
}
\usepackage{float} % Good placement for float objects
\usepackage[export]{adjustbox} % Positioning figures
\usepackage{amsmath,amsthm,amssymb} % American Mathematics Society facilities

% --------------------------------------------------------------
%                       Header Definition
% --------------------------------------------------------------
\newcommand{\createHeader}[5]{
	\noindent
	\normalsize\textbf{#2} \hfill \textbf{#1}\\
	\normalsize\textbf{#3} \hfill \textbf{\today}\vspace{10pt}
	\centerline{\Large \textbf{#5}}\vspace{1pt}
	\centerline{\small #4}\vspace{20pt}}


% --------------------------------------------------------------
%                         Languages
%			Change also the exercise environment
% --------------------------------------------------------------
\usepackage[english]{babel} % Multilingual support [english]
%\usepackage[french]{babel} % Multilingual support [french]
%\usepackage[brazilian]{babel} % Multilingual support [pt-BR]


% --------------------------------------------------------------
%                         Fonts
% --------------------------------------------------------------
\usepackage{lmodern} % Good looking T1 font
%\usepackage{mathpazo} % Hermann Zapf's Palatino font
%\usepackage{kpfonts} % Kepler font
%\usepackage{mathptmx} % Times New Roman Like Font
%\usepackage{eulervm} %  AMS Euler (eulervm) math font.

% --------------------------------------------------------------
%                       Begin Document
% --------------------------------------------------------------

\begin{document}
	\createHeader{École Polytechnique}{Rock}{LAN484jANG}{Alexandre Ribeiro João Macedo}{HW - A Day In The Life}
	
\begin{linenumbers}
A Day In The Life is a Beatles song that first appeared on their album Sgt Pepper's Lonely Hearts Club Band written mainly by John Lennon and with a contribution by Paul McCartney in the middle section. According to the website \href{https://www.beatlesbible.com/songs/a-day-in-the-life/}{Beatles Bile}, this song is the climax of the aforementioned album when the Beatles were in th peak of their creative powers and it was a song that entered history.

The lyrics were inspired by newspaper article that included the death of Tara Brown, the heir of Guinness brewery, and the huge amount of holes in the road in Blackburn, Lancashire. By knowing that, we can create a direct connection between the title of the song and its lyrics. The everyday events we see in the news as just another "day in the life".

Another aspect that should be mentioned about this song is its references to drugs that made it be banned by BBC due to the line "I'd love to turn you on". According to Ian MacDonald, this song was strongly influenced by John Lennon's LSD experiences.

One of the most remarkable parts of this song is the orchestra part where the musicians were instructed to improvise over the segment. For McCartney it should create a build-up from nothing to something like the end of the world. For the listener it is easy to see that he achieved his goal.

It is not by chance that this become one landmark in pop music. The musical arrangements developed were really complex but considering it was made by the Beatles, it is a common thing. 
\end{linenumbers}

\end{document}
