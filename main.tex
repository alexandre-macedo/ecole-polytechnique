% --------------------------------------------------------------
%                         Template DM
% --------------------------------------------------------------
\documentclass[10pt]{article} %draft = show box warnings
\usepackage[a4paper, total={6.5in,10.2in}]{geometry} % Flexible and complete interface to document dimensions

\usepackage[utf8]{inputenc} % Accept different input encodings [utf8]
\usepackage[T1]{fontenc}    % Standard package for selecting font encodings
\usepackage{lmodern} % Good looking T1 font
% --------------------------------------------------------------
%                         Other Fonts
% --------------------------------------------------------------
%\usepackage{mathpazo} % Hermann Zapf's Palatino font
%\usepackage{kpfonts} % Kepler font
%\usepackage{mathptmx} % Times New Roman Like Font
%\usepackage{eulervm} %  AMS Euler (eulervm) math font.

% --------------------------------------------------------------
%                         Packages
% --------------------------------------------------------------
\usepackage[english]{babel} % Multilingual support
\usepackage{graphicx} % Enhanced support for graphics
\usepackage{float} % Improved interface for floating objects
\usepackage{amsmath,amsthm,amssymb} % American Mathematics Society facilities
\usepackage{hyperref}
\usepackage{listings}
\usepackage{color}
\usepackage{verbatim}

% --------------------------------------------------------------
%                       Exercise Env
% --------------------------------------------------------------
\newtheoremstyle{problemstyle}  % <name>
{3pt}                   % <space above>
{3pt}                   % <space below>
{\normalfont}           % <body font>
{}                      % <indent amount}
{\bfseries}             % <theorem head font>
{\normalfont\bfseries\hspace{0.2 cm}}  % <punctuation after theorem head>
{.5em}                  % <space after theorem head>
{}                      % <theorem head spec (can be left empty, meaning `normal')>
\theoremstyle{problemstyle}

% Change Language
\newtheorem{exercise}{Task}	% Englsih
%\newtheorem{exercise}{Exercice} 	% French
%\newtheorem{exercise}{Exercício}	% Pt-BR

% Change counter type
%\renewcommand{\theexercise}{\Roman{exercise}. } % Exercise I)
\renewcommand{\theexercise}{\arabic{exercise}.} % Exercise 1.
%\renewcommand{\theexercise}{\Alph{exercise}.} % Exercise A.
%\renewcommand{\theexercise}{\alph{exercise}.} % Exercise a.

% --------------------------------------------------------------
%                       Listing conf
% --------------------------------------------------------------
\definecolor{codegreen}{rgb}{0,0.6,0}
\definecolor{codegray}{rgb}{0.5,0.5,0.5}
\definecolor{codepurple}{rgb}{0.58,0,0.82}
\definecolor{backcolour}{rgb}{0.95,0.95,0.95}
\definecolor{backcolourOld}{rgb}{0.95,0.95,0.92}
 
\lstset{
  backgroundcolor=\color{backcolour},	% choose the background color;
  basicstyle=\footnotesize,				% the size of the fonts that are used for the code
  keywordstyle=\color{magenta},			% keyword style
  stringstyle=\color{codepurple},		% string literal style
  commentstyle=\color{codegreen},		% comment style
  numberstyle=\tiny\color{codegray},	% the style that is used for the line-numbers
  rulecolor=\color{black},				% if not set, the frame-color may be changed on line-breaks within not-black text (e.g. comments (green here))
  breakatwhitespace=false,				% sets if automatic breaks should only happen at whitespace
  breaklines=true,						% sets automatic line breaking
  frame=single,							% adds a frame around the code
  keepspaces=true,						% keeps spaces in text, useful for keeping indentation of code (possibly needs columns=flexible)
  numbers=left,							% where to put the line-numbers; possible values are (none, left, right)
  numbersep=5pt,						% how far the line-numbers are from the code
  showspaces=false,						% show spaces everywhere adding particular underscores; it overrides 'showstringspaces'
  showstringspaces=true,				% underline spaces within strings only
  showtabs=false,						% show tabs within strings adding particular underscores
  stepnumber=1,							% the step between two line-numbers. If it's 1, each line will be numbered
  tabsize=2,							% sets default tabsize to 2 spaces
  captionpos=t							% caption position
}




% --------------------------------------------------------------
%                       Custom commands, counters
% --------------------------------------------------------------
\renewcommand{\lstlistingname}{Algorithm}
\renewcommand*{\O}{\mathcal{O}}

% --------------------------------------------------------------

\begin{document}

% --------------------------------------------------------------
%                       Header
% --------------------------------------------------------------
\noindent
\normalsize\textbf{Conception et analyse d'algorithmes} \hfill \textbf{École Polytechnique}\\
\normalsize\textbf{INF 421} \hfill \textbf{\today}\vspace{20pt}
\centerline{\Large Programming Project P1 – X2015}\vspace{5pt}
\centerline{\Large \textbf{From ADN to formation of proteins : how to align sequences ?}}\vspace{3pt}
\centerline{Project proposed by Marie Albenque -- \texttt{ albenque@lix.polytechnique.fr}}\vspace{13pt}
\centerline{Lucas Lugão Guimarães -- \texttt{lucas.lugao-guimaraes@polytechnique.edu}}
\centerline{Alexandre Ribeiro João Macedo --  \texttt{alexandre.macedo@polytechnique.edu}}\vspace{20pt}
% --------------------------------------------------------------

%% http://www.lix.polytechnique.fr/~albenque/INF421-2016/Alignment_Project.pdf

\begin{exercise} % Task 1
The longest common subsequence (LCS) problem can be solved for two sequences $s=s_1\dots s_n$ and $t=t_1\dots t_m$ in a naive way by considering all the $2^{n}$ subsequences of $s$ and determining if it is a subsequence of $t$. The last step can be done with a complexity of $\O(m)$, which leaves the naive algorithm with a complexity of $\O(2^{n}m)$.
\end{exercise}

\begin{exercise} % Task 2
In order to solve the LCS in a more efficient way, dynamic programming (DP) can be used. We denote $LCS(s,t)$ the solution of the problem given the inputs $s=s_1\dots s_n$ and $t=t_1\dots t_m$. In the case where $s_m = t_n = u$, the solution will be $u$ append to the end of the solution of $LCS(\hat{s}=s_1\dots s_{n-1}, \hat{t}=t_1\dots t_{m-1})$. In the case where $s_m \neq t_n$, the solution will be the longest sequence between $LCS(\hat{s}=s_1\dots s_{n-1}, t)$ and $LCS(s, \hat{t}=t_1\dots t_{m-1})$. In a more straight-forward way, this problem can be solved as described by \cite{jones}. One implementation of this DP in Java is given by the algorithm \ref{task2} whose complexity is $\O(nm)$.
\begin{comment}
This solution consists of filling two matrices $A_{n+1,m+1}$ and $B_{n+1,m+1}$, following the given rules: initialize the first line and the first column of $A$ in zeros, fill this matrix according to 
\begin{align*}
A_{i,j}
\end{align*}
\end{comment}

\lstinputlisting[caption={Longest common subsequence}, label=task2, language=java]{Solution.java}
\end{exercise}

\begin{exercise} % Task 3
The editing distance given two sequences $s=s_1\dots s_n$ and $t=t_1\dots t_m$ can be thought as being equivalent to adding hyphens in $s$ and $t$ in such a way that we minimize the numbers of mentioned hyphens added to the number of letters that $s$ and $t$ differ if we consider that adding an hyphen to a sequence, lets say $s$, is equivalent to an insertion in $s$ or a deletion in $t$ and vice-versa. The difference between the letters can be interpreted as a transforming in $s$ or $t$. It is important to notice that given an editing distance there is more than one way of insert, delete and transform the characters to find the alignment and for each choice we have a different alignment with the same editing distance.

The algorithm \ref{task3} computes and displays one optimal alignment.
\lstinputlisting[caption={Optimal alignment}, label=task3, language=java]{Solution.java} 
\end{exercise}

\begin{exercise} % Task 4
	content...
\end{exercise}

\begin{exercise} % Task 5
	content...
\end{exercise}

\begin{exercise} % Task 6
	content...
\end{exercise}

\begin{exercise} % Task 7
	content...
\end{exercise}

\begin{exercise} % Task 8
	content...
\end{exercise}

\begin{exercise} % Task 9
	content...
\end{exercise}

\begin{thebibliography}{3}
	
	\bibitem{jones} 
	Neil C. Jones and Pavel A. Pevzner. 
	\textit{An Introduction to Bioinformatics Algorithms}. 
	MIT Press, 2004
	
	\bibitem{durbin} 
	Richard Durbin, Sean R. Eddy, Anders Krogh and Graeme Mitchison. 
	\textit{Biological Sequence Analysis: Probabilistic Models of Proteins and Nucleic Acids}. 
	Cambridge University Press, 1998
	
\end{thebibliography}

\end{document}
