% 

% --------------------------------------------------------------
%                         Template
% --------------------------------------------------------------

\documentclass[12pt]{article} %draft = show box warnings
\usepackage[utf8]{inputenc} % Accept different input encodings [utf8]
\usepackage[T1]{fontenc}    % Standard package for selecting font encodings
\usepackage{lmodern} % Good looking T1 font
\usepackage[a4paper, total={6.5in,10.2in}]{geometry} % Flexible and complete interface to document dimensions
\renewcommand{\baselinestretch}{1.0}

% --------------------------------------------------------------
%                         Other Fonts
% --------------------------------------------------------------
%\usepackage{mathpazo} % Hermann Zapf's Palatino font
%\usepackage{kpfonts} % Kepler font
%\usepackage{mathptmx} % Times New Roman Like Font
%\usepackage{eulervm} %  AMS Euler (eulervm) math font.

% --------------------------------------------------------------
%                         Package
% --------------------------------------------------------------
\usepackage[french]{babel} % Multilingual support [french]
\usepackage{enumerate} % Enumerate with redefinable labels
\usepackage{graphicx} % Enhanced support for graphics
\usepackage[makeroom]{cancel} % Place lines through maths formulae
\usepackage{booktabs} % Publication quality tables
\usepackage{braket} % Dirac bra-ket and set notations
\usepackage{epstopdf} % Convert EPS to ‘encapsulated’ PDF using Ghostscript
\usepackage{bbm} % "Blackboard-style" cm fonts
\RequirePackage{epsfig} % Include Encapsulated PostScript
\usepackage{float} % Improved interface for floating objects
\usepackage{amsmath,amsthm,amssymb} % American Mathematics Society facilities

\usepackage{stmaryrd} % St Mary Road symbols for theoretical computer science.
\usepackage{xfrac} % Create inline fraction
\usepackage{hyperref} % Create hyperf
\usepackage{secdot} % Put dots at the end of a section



%------------------- tikz --------------------------------------
\usepackage{tikz,bm,color}
\usetikzlibrary{shapes,arrows}
\usetikzlibrary{decorations}
\usetikzlibrary{decorations.pathreplacing,angles,quotes}
\usetikzlibrary{shapes.geometric,arrows,positioning}
\usetikzlibrary{calc}

% --------------------------------------------------------------
%                         Custom commands
% --------------------------------------------------------------
\newcommand*{\1}{\mathbbm{1}}
\newcommand*{\E}{\mathbb{E}}
\newcommand*{\Pb}{\mathbb{P}}
\newcommand*{\N}{\mathbb{N}} 
\newcommand*{\Z}{\mathbb{Z}}
\newcommand*{\car}{\text{card}}
\renewcommand*{\Re}{\operatorname{Re}}
\renewcommand*{\Im}{\operatorname{Im}}
\newcommand*\Laplace{\mathop{}\!\mathbin\bigtriangleup}
\newcommand*{\grad}{\nabla}

% --------------------------------------------------------------
%                         Exercise Environment
% --------------------------------------------------------------
\renewcommand{\thesection}{\Roman{section}}
\sectiondot{subsection}
\renewcommand{\subsubsection}{
	\pagebreak[2]
	\refstepcounter{subsubsection}
	\noindent
	\textbf{\thesubsubsection.}
}
% --------------------------------------------------------------

\begin{document}
	% --------------------------------------------------------------
	%                       Title  Header
	% --------------------------------------------------------------
	\noindent
	\normalsize\textbf{Approximation numérique et optimisation} \hfill \textbf{École Polytechnique}\\
	\normalsize\textbf{MAP 411} \hfill \textbf{\today}\vspace{20pt}
	\centerline{\Large Projet MAP411 – X2015}\vspace{5pt}
	\centerline{\Large \textbf{Applications de méthodes d'optimisation à la résolution des EDP}}\vspace{3pt}
	\centerline{Sujet proposé par Nicolas Augier -- \texttt{nicolas.augier@ens-cachan.fr}  }\vspace{13pt}
	\centerline{Lucas Lugão Guimarães -- \texttt{lucas.lugao-guimaraes@polytechnique.edu}  }
	\centerline{Alexandre Ribeiro João Macedo --  \texttt{alexandre.macedo@polytechnique.edu}}\vspace{20pt}
	% --------------------------------------------------------------
	
	\section{Méthodes de gradient}
	\subsection{Gradient à pas constant}
	\subsubsection{} %%I.1.1
	La figure \ref{fig:I11} montre les lignes de niveau de la fonction de Rosenbrock,
	\begin{align} \label{eq:rosenbrock}
	J(u_1,u_2)=(u_1-1)^2+100(u_1^2-u_2)^2 \text{,}
	\end{align}
	dans le rectangle $(u_1, u_2) \in [-1.5, 1.5] \textmd{ x } [0.5, 1.5]$.
	\begin{figure}[ht]
		\begin{center}
	\includegraphics[width=5cm]{nada.png}
		\end{center}
		\caption{Lignes de niveau de $J$.}
		\label{fig:I11}
	\end{figure}
	
	\subsubsection{} %% I.1.2
	On applique la méthode de gradient a pas constant $u_{k+1}=u_k-\rho \grad J(u_k)$ ou, pour $ u_k= (u_{k1},u_{k2})$, on a $\grad J(u_k) = ( 400u_{k1}^3 + 2u_{k1}  - 400u_{k1}u_{k2} - 2, 200u_{k1}^2 - 200u_{k2})$. Pour les valeurs numériques $u_0=(1, 0)$ et $\rho = 0.002$, et le critère d'arrêt $J(u_{k+1}) < 10^{-3}$, on a besoin de \textbf{$12312$} itérations. La figure \ref{fig:I12} montre les lignes de niveau de l'équation \ref{eq:rosenbrock} et le parcours du méthode avant l'arrête.
	\begin{figure}[ht]
		\begin{center}
	\includegraphics[width=5cm]{nada.png}
		\end{center}
		\caption{Parcours du méthode de gradient à pas constant.}
		\label{fig:I12}
	\end{figure}
	
	\subsubsection{} %% I.1.3
	Si on applique le méthode pour $\rho = 0.0045$ et pour $\rho = 0.01$, on a que le méthode diverge. 
	
	\subsection{Gradient optimal}
	\subsubsection{}
	Soit $J(u)=\frac{1}{2} \langle Au, u \rangle - \langle b, u \rangle$. On veut trouver ${\rho_k=\textrm{argmin}_{\rho \geq 0} J(u_k- \rho \grad J(u_k))}$. Soit ${r_k= \grad J(u_k) = Au_k-b}$, on a
	\begin{align} \label{eq:gradopt}
	\begin{split}
	J(u_k - \rho \grad J(u_k)) &= J(u_k - \rho r_k) \\ 
	&=  \frac{1}{2} \langle A(u_k - \rho r_k), u_k - \rho r_k \rangle - \langle b, u_k - \rho r_k \rangle \\
	&=\frac{1}{2} \langle Au_k, u_k \rangle - \langle b, u_k \rangle  + \frac{1}{2} \langle Au_k, - \rho r_k \rangle + \frac{1}{2} \langle - \rho A r_k, u_k \rangle + \frac{1}{2} \langle - \rho A r_k, - \rho r_k \rangle  \\
	&  - \langle b, - \rho r_k \rangle \\
    &= J(u_k) - \frac{\rho}{2} [\langle A u_k, r_k \rangle + \langle A r_k, u_k \rangle]+\frac{\rho^2}{2} \langle A r_k, r_k \rangle + \rho \langle b,  r_k \rangle \\
    &=J(u_k)  +\frac{\rho^2}{2} \langle A r_k, r_k \rangle + \rho [\langle b,  r_k \rangle -\langle A u_k, r_k \rangle] \\
    &=J(u_k)  +\frac{\rho^2}{2} \langle A r_k, r_k \rangle + \rho [\langle b,  r_k \rangle -\langle r_k - b, r_k \rangle] \\
      &=J(u_k)  +\frac{\rho^2}{2} \langle A r_k, r_k \rangle + \rho \langle r_k, r_k \rangle \textrm{.}
	\end{split}
	\end{align}
	Alors, la valeur de $\rho$ que minimise l'équation \ref{eq:gradopt} est
    \begin{align*}
    \rho_k=\frac{\langle r_k, r_k \rangle}{\langle A r_k, r_k \rangle} \textrm{.}
    \end{align*}
    
    Pour les valeur
    \begin{align*}
    A=
    \begin{pmatrix}
    1 & 0 \\
    0 & \lambda
    \end{pmatrix} \qquad \text{ et } \qquad b=
    \begin{pmatrix}
    1 \\
    1
    \end{pmatrix}
    \end{align*}
    On a 
\end{document}